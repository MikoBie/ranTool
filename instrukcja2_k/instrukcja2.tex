\documentclass[12pt, crop,varwidth,border=3pt,convert={density=300,outext=.png}]{standalone}
\usepackage[polish]{babel}
\hyphenpenalty=10000
\usepackage{xcolor}


\begin{document}
\textcolor{white}{Na samym początku prosilibyśmy Cię o wyobrażenie sobie wyniku rzutu zwykłą monetą, dla której prawdopodobieństwo wyrzucenia zarówno orła jak i reszki jest równe. Za moment na ekranie komputera zacznie wyświetlać się czerwony kwadrat. Za każdym razem gdy się wyświetli prosilibyśmy Cię o~podjęcie decyzji czy w wyobrażonym rzucie monetą wypadła reszka czy orzeł. Naciśnij na klawiaturze O, jeśli był to orzeł albo R jeśli była to reszka. Przez następne 10 minut prosilibyśmy Cię, żebyś za każdym razem gdy wyświetli się czerwony kwadrat podjęła tę decyzję i nacisnęła odpowiedni klawisz.}
\newline

\textcolor{white}{Usiądź teraz sobie w wygodnej pozycji oraz połóż palce wskazujący i~środkowy dominującej ręki na odpowiednich przyciskach.}

\begin{center}
\textcolor{white}{\Large Naciśnij spację, żeby przejść dalej.}
\end{center}
\end{document}